\documentclass[10pt, a4paper, oneside]{basestyle}
\usepackage[Mathematics]{semtex}

%%%% Shorthands.

%%%% Title and authors.

\title{%
\textdisplay{%
Calculations for the second-order zonal harmonic%
}%
}
\author{Pascal~Leroy (phl) \& Robin~Leroy (eggrobin)}
\begin{document}
\maketitle
\noindent
Notations:\[
\vr=\begin{pmatrix}
x \\ y \\ z
\end{pmatrix} \text{, }
\norm{\vj}=1 \text{, }
r = \norm{\vr}\text.
\]
For oblateness along the $z$-axis, the potential is
\[
\frac{J_2}{2r^5}\pa{3z^2 - r^2}\text.
\]
For oblateness along $\vj$,
\[
U\of\vr = \frac{J_2}{2r^5}\pa{3\pascal{\scal{\vr}{\vj}}^2 - r^2}\text.
\]
Differentiating,
\[
\deriv \vr U = \frac{J_2}{2}\pa{-\frac{5}{r^6} \deriv \vr r
                   \pa{3\pascal{\scal\vr\vj}^2 - r^2}
               + \frac{1}{r^5}\pa{3\derivop \vr \pascal{\scal\vr\vj}^2 -
                   2 r \deriv \vr r}}\text.
\]
Recall that $\deriv \vr r = \frac{\vr}{r}$,
\begin{align*}
\deriv \vr U &= \frac{J_2}{2}\pa{-\frac{5\vr}{r^7}
                    \pa{3\pascal{\scal\vr\vj}^2 - r^2}
                + \frac{1}{r^5}\pa{3\derivop \vr \pascal{\scal\vr\vj}^2 -
                    2 \vr}} \\
             &= \frac{J_2}{2}\pa{-\frac{15\vr}{r^7}
                    \pascal{\scal\vr\vj}^2 + \frac{3\vr}{r^5}
                + \frac{3}{r^5}3\derivop \vr \pascal{\scal\vr\vj}^2}\text.
\end{align*}
With $\derivop \vr \pascal{\scal\vr\vj}^2 =
  2\pascal{\scal\vr\vj}\deriv\vr{\pascal{\scal\vr\vj}} = 2\pascal{\scal\vr\vj}\vj$,
\begin{align*}
\deriv \vr U &= \frac{J_2}{2}\pa{-\frac{15\vr}{r^7}
                    \pascal{\scal\vr\vj}^2 + \frac{3\vr}{r^5}
                + \frac{6\vj}{r^5}\pascal{\scal\vr\vj}} \\
             &= \frac{3J_2}{2r^5}\pa{2\vj\pascal{\scal\vr\vj}
                + \vr \pa{1 - \frac{5\pascal{\scal\vr\vj}^2}{r^2}}}\text.
\end{align*}
Note that this is invariant under $\vj\mapsto -\vj$ (but not under $\vr\mapsto -\vr$).
\vfill
\pagebreak

\noindent
Implementation:
{\small%
\begin{verbatim}

// If j is a unit vector along the axis of rotation, and r a vector from the
// center of |body| to some point in space, the acceleration computed here is:
//
//   -(J₂ / (μ ‖r‖⁵)) (3 j (r.j) + r (3 - 15 (r.j)² / ‖r‖²) / 2)
//
// Where ‖r‖ is the norm of r and r.j is the inner product.  It is the
// additional acceleration exerted by the oblateness of |body| on a point at
// position r.  J₂, J̃₂ and J̄₂ are normally positive and C̃₂₀ and C̄₂₀ negative
// because the planets are oblate, not prolate.  Note that this follows IERS
// Technical Note 36 and it differs from
// https://en.wikipedia.org/wiki/Geopotential_model which seems to want J̃₂ to be
// negative.
template<typename Frame>
FORCE_INLINE Vector<Quotient<Acceleration, GravitationalParameter>, Frame>
Order2ZonalAcceleration(OblateBody<Frame> const& body,
                        Displacement<Frame> const& r,
                        Exponentiation<Length, -2> const& one_over_r_squared,
                        Exponentiation<Length, -3> const& one_over_r_cubed) {
  Vector<double, Frame> const& axis = body.polar_axis();
  Length const r_axis_projection = InnerProduct(axis, r);
  auto const j2_over_r_fifth =
      body.j2_over_μ() * one_over_r_cubed * one_over_r_squared;
  Vector<Quotient<Acceleration,
                  GravitationalParameter>, Frame> const axis_effect =
      (-3 * j2_over_r_fifth * r_axis_projection) * axis;
  Vector<Quotient<Acceleration,
                  GravitationalParameter>, Frame> const radial_effect =
      (j2_over_r_fifth *
           (-1.5 +
            7.5 * r_axis_projection *
                  r_axis_projection * one_over_r_squared)) * r;
  return axis_effect + radial_effect;
}
\end{verbatim}%
}
\end{document}