\documentclass[10pt, a4paper, oneside]{basestyle}
\usepackage[Mathematics]{semtex}

%%%% Shorthands.

%%%% Title and authors.

\newcommand{\point}[1]{\mathrm{#1}}
\newcommand{\bipoint}[2]{\overrightarrow{\point #1 \point #2}}
\newcommand{\straightline}[2]{\point #1 \point #2}
\newcommand{\plane}[3]{\point #1 \point #2 \point #3}
\newcommand{\squarenorm}[1]{\scal{#1}{#1}}

\title{%
\textdisplay{%
Documentation for the hiding computations in Planetarium%
}%
}
\author{Pascal~Leroy (pleroy), typeset by Robin~Leroy (eggrobin)}
\begin{document}
\maketitle
$\point A$ and $\point B$ are the extremities of the segment; $\point C$ is the
centre of the sphere; $R$ is the radius of the sphere; $\point K$ is the
location of the camera.

For simplicity we will do most of our analysis in the plane $\plane KAB$.
We will use $\tuple{\bipoint KA, \bipoint KB}$ as a basis of that plane.
Let $\point H$ be the orthogonal projection of $\point C$ on $\plane KAB$. Define
$\ga$, $\gb$ to be its coordinates in $\plane KAB$:\[
\bipoint KH = \ga \bipoint KA + \gb \bipoint KB\text.
\]
Note that $\bipoint KH = \bipoint KC + \bipoint CH$. By its definition,
$\bipoint CH$ is orthogonal to both $\bipoint KA$ and $\bipoint KB$:\begin{align*}
\scal{\bipoint KA}{\bipoint CH} &= 0\\
\scal{\bipoint KB}{\bipoint CH} &= 0\text,
\end{align*}
or:
\begin{align*}
\scal{\bipoint KA}{\bipoint KH} &= \scal{\bipoint KA}{\bipoint KC}\\
\scal{\bipoint KB}{\bipoint KH} &= \scal{\bipoint KB}{\bipoint KC}\text.
\end{align*}
Expanding $\bipoint KH$ gives a linear system of two equations with two unknowns:
\begin{align*}
\ga \scal{\bipoint KA}{\bipoint KA} + \gb \scal{\bipoint KA}{\bipoint KB}
    &= \scal{\bipoint KA}{\bipoint KC}\\
\ga \scal{\bipoint KA}{\bipoint KB} + \ga \scal{\bipoint KB}{\bipoint KB}
    &= \scal{\bipoint KB}{\bipoint KC}\text.
\end{align*}
The determinant of this system is:\[
D = \pascal{\squarenorm{\bipoint KA}} \pascal{\squarenorm{\bipoint KB}}
    - \pascal{\scal{\bipoint KA}{\bipoint KB}}^2\text,
\]
which is non-zero if and only if $\point A \neq \point B$. The solutions are thus
\begin{align*}
\ga &= \frac
  {\pascal{\squarenorm{\bipoint KB}} \pascal{\scal{\bipoint KA}{\bipoint KC}} -
   \pascal{\scal{\bipoint KA}{\bipoint KB}} \pascal{\scal{\bipoint KB}{\bipoint KC}}}
  {D}\\
\gb &= \frac
  {\pascal{\squarenorm{\bipoint KA}} \pascal{\scal{\bipoint KB}{\bipoint KC}} -
   \pascal{\scal{\bipoint KA}{\bipoint KB}} \pascal{\scal{\bipoint KA}{\bipoint KC}}}
  {D}\text.
\end{align*}
Now that $\bipoint KH$ is determined we can compute
$\bipoint CH = \bipoint KH - \bipoint KC$ and $\scal{\bipoint CH}{\bipoint CH}$.
If $\squarenorm{\bipoint CH} \geq R^2$, the sphere is either tangent to
the plane $\plane KAB$ or doesn't intersect it. Thus, there is no hiding.

Let's look at a figure in the plane $\plane KAB$ when the sphere intersects
that plane, considering first the situation where the intersection circle is
between the camera and the segment.
$r$ is the radius of the circle, it is such that
$r^2 = R^2 - \squarenorm{\bipoint CH}$. $\point P$ is a point where a line
going through $K$ is tangent to the circle.
$\point Q$ is the point where $\straightline KP$ intersects $\straightline AB$; it may be
between $\point K$ and $\point P$ or behind $\point P$.

We need to find $\point P$. It is defined by two equations:
\begin{align*}
\squarenorm{\bipoint PH} &= r^2\\
\scal{\bipoint PH}{\bipoint KP} &= 0\text.
\end{align*}

The second equation may be rewritten:
\begin{align*}
\scal{\bipoint PH}{\pascal{\bipoint KH + \bipoint HP}} &= 0 \\
\scal{\bipoint PH}{\bipoint KH} &= r^2\text.
\end{align*}
Let $\gg$ and $\gd$ be the coordinates of $\bipoint PH$ in $\plane KAB$:\[
\bipoint PH = \gg {\bipoint KA} + \gd {\bipoint KB}\text.
\]
The second equation is linear:\[
\gg\scal{\bipoint KA}{\bipoint KH} + \gd\scal{\bipoint KB}{\bipoint KH} = r^2\text.
\]
Thus:\[
\gg = \frac{r^2 - \gd \scal{\bipoint KB}{\bipoint KH}}{\scal{\bipoint KA}{\bipoint KH}}\text.
\]
Note that $\scal{\bipoint KA}{\bipoint KH}$ and $\scal{\bipoint KB}{\bipoint KH}$ cannot both be $0$
unless $\point K$ is on $\straightline AB$.

The first equation is quadratic:\[
\pa{\gg\bipoint KA + \gd \bipoint KB}^2 = r^2\text.
\]
Pluging the value of $\gg$ above we get:
\begin{align*}
\gd^2\pa{
 \pascal{\squarenorm{\bipoint KB}} \pascal{\scal{\bipoint KA}{\bipoint KH}}^2 +
 2\pascal{\scal{\bipoint KA}{\bipoint KB}}
  \pascal{\scal{\bipoint KA}{\bipoint KH}}
  \pascal{\scal{\bipoint KB}{\bipoint KH}} +
 \squarenorm{\bipoint KA} \pascal{\scal{\bipoint KB}{\bipoint KH}}^2} &+\\
2\gd r^2\pa{
 \pascal{\scal{\bipoint KA}{\bipoint KB}}
 \pascal{\scal{\bipoint KA}{\bipoint KH}} -
 \pascal{\squarenorm{\bipoint KA}} \pascal{\scal{\bipoint KB}{\bipoint KH}}} &+\\
r^2\pa{
 r^2\pascal{\squarenorm{\bipoint KA}} -
 \pascal{\scal{\bipoint KA}{\bipoint KH}}^2} &= 0\text.
\end{align*}
This equation always has two solutions because the sphere intersects $\plane KAB$.
Having determined $\bipoint PH$, we can find $\point Q$. $\point Q$ is on the line
$\straightline AB$, thus:\[
\bipoint AQ = \gl \bipoint AB\text,
\] which can be written \[
\bipoint KQ - \bipoint KA = \gl \bipoint AB\text.
\]
Noting that $\bipoint KQ$ is orthogonal to $\bipoint PH$ we have\[
-\scal{\bipoint KA}{\bipoint PH} = \gl \scal{\bipoint AB}{\bipoint PH}\text{, or, }
\gl = -\frac{\scal{\bipoint KA}{\bipoint PH}}{\scal{\bipoint AB}{\bipoint PH}}\text.
\]
It is possible though that $point Q$ would be between $\point K$ and $\point P$, in
which case the segment would not intersect the cone. To determine if this is the case we write:\[
\bipoint KP = \bipoint KH -\bipoint PH = \pa{\ga-\gg}\bipoint KA + \pa{\gb-\gd}\bipoint KB\text.
\]
The line $\straightline AB$ is the set of points of the form $\gh\bipoint KA + \pa{1-\gh}\bipoint KB$
so $\point P$ is in front of $\straightline AB$ if and only if\[
\ga-\gg+\gb-\gd < 1\text.
\]

To complete the analysis we need to compute the intersection of the sphere (not the cone) with the
line $\straightline AB$. $\point Q$ is on the sphere, thus\[
\squarenorm{\bipoint CQ} = R^2\text.
\]
It is also on the line $\straightline AB$ thus\[
\bipoint KQ = \bipoint KA + \gm \bipoint AB\text.
\]
We have:\[
\bipoint CQ = \bipoint KQ - \bipoint QC =
  \bipoint KA + \gm \bipoint KC =
  \bipoint CA + \gm \bipoint AB\text,
\]
and therefore\[
R^2 = \pascal{\bipoint CA + \gm \bipoint AB}^2\text,
\]
meaning that $\gm$ is a solution of\[
\gm \squarenorm{\bipoint AB} + 2 \gm \scal{\bipoint CA}{\bipoint AB}
+ \squarenorm{\bipoint CA} - R^2 = 0\text.
\]
Depending on the location of the segment with respect to the sphere, there can be $0$, $1$, or $2$
intersections.

If we take the union of the values of $\gl$ and $\gm$ and order them, it is straightforward to
find the visible segments. Remember that $0<\gl,\gm<1$ for points that are in the segment
$\straightline AB$.
\end{document}
